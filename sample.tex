\documentclass[12pt]{article}

\usepackage{amsmath}
\usepackage{amssymb}
\usepackage{amsthm}
\usepackage{enumerate}

%Gummi|065|=)
\title{\textbf{Welcome to Gummi 0.6.5}}
\author{Alexander van der Mey \and 
		Wei-Ning Huang \and
		Dion Timmermann}
%\date{}
\begin{document}

\maketitle

\newtheorem{lemma}{Lemma}
\newtheorem{theorem}[lemma]{Theorem}

%\chapter{This is a chapter}

\section{Before you start}

Hahaha\\

The sine function is $\sin^{100/3} x$.
Now we have
\[
\frac{\sin^2 x+\cos^2 x}{3}=\frac{1}{3}.
\]

$\{a^b+a_{i,j}\}$

\[
(\frac{1}{3}+2).
\]


\[
\left(\frac{1}{3}+2\right).
\]


$\alpha$, $\beta$, $\gamma$, $\delta$, $\epsilon$,
$\leq$, $\geq$, $\equiv$, $\in$, $\subseteq$, $\notin$,
$\not\subseteq$,
$\subsetneq$, $\cup$, $\cap$, $\land$, $\lor$, $\forall$,
$\exists$,
$\rightarrow$,
$\leftarrow$
$\Rightarrow$,
$\lceil 3.6\rceil$, $\lfloor 3.7\rfloor$,
$\text{poly}(x)$.

-\\
15--20\\
There are so many people---590.\\
$\ldots$, $\cdots$, $a\cdot b$, $a\times b$, $a+b-c$

\begin{eqnarray}
A&=&B+C+D\\
&=&C\label{myequation}\\
&=&D
\end{eqnarray}

Equation~(\ref{myequation})

\begin{lemma}
We have a may-lab.
\end{lemma}

\begin{theorem}\label{mytheorem}
Every closed and bounded set in $\mathbb{R}^n$
is compact.
\end{theorem}
\begin{proof}
This is my proof for the theorem.
\end{proof}


Theorem~\ref{mytheorem}

Our lab members include
\begin{itemize}
\item Small rain,
\item Big rain, and
\item Cat bro.
\end{itemize}


Our lab members include
\begin{enumerate}
\item Small rain,
\item Big rain, and
\item Cat bro.
\end{enumerate}

Our lab members include
\begin{enumerate}[(i)]
\item\label{myitem} Small rain,
\item Big rain, and
\item Cat bro.
\end{enumerate}

Item~(\ref{myitem}) says that \ldots


You are now using Gummi 0.6.5. Many new exciting features have been added to the 0.6 series. The document editor is now a tabbed instance, allowing multiple documents to be worked on simultaneously. Using the new projects menu, you can group files together for easy access. 

Support for two high-level {\LaTeX} building systems, \emph{rubber}\footnote{https://launchpad.net/rubber/} \& \emph{latexmk}\footnote{http://www.phys.psu.edu/{\textasciitilde}collins/software/latexmk-jcc/} has been added to this release as well. Your preferred typesetter can be configured through the Compilation tab in the Preferences menu. Typesetters that are not installed on your system will not be selectable. 

Added for your viewing convenience is a continuous preview mode for the PDF. This mode is enabled by default, but can also be disabled through the \emph{(View $\rightarrow$ Page layout in preview)} menu. Complementary to this feature is SyncTeX integration, which allows you to synchronize the position in your editor with the PDF preview. 

\section{Feedback}
We hope you will enjoy using this release as much as we enjoyed creating it. If you have comments, suggestions or wish to report an issue you are experiencing - contact us at: \emph{http://gummi.midnightcoding.org}.

\section{One more thing}
If you are wondering where your old default text is; it has been stored as a template. The template menu can be used to access and restore it. 

\end{document}
