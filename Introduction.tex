\documentclass[12pt]{article}

\usepackage{amsmath}
\usepackage{amssymb}
\usepackage{amsthm}
\usepackage{enumerate}

%Gummi|065|=)
\title{\textbf{XXXX}}
%\author{Alexander van der Mey \and 
%		Wei-Ning Huang \and
%		Dion Timmermann}

\date{}
\begin{document}

\maketitle

\section{Introduction}
A graph is an ordered pair $G=(V,E)$ consisting of a finite nonempty set $V$ of vertices and a set $E$ of edges, where each edge is an unordered pair of vertices. A dominating set of a graph $G=(V,E)$ is a set $D \subseteq V$ such that each vertex not in $D$ has at least one neighbor in $D$. A paired-dominating set is a dominating set whose induced subgraph contains at least one perfect matching \cite{lamport1}.


Raz and Safra prove that the dominating set problem has no polynomial-time can implement an approximation algorithms better than $C\log|V|$  \cite{lamport2}.

Ching-Chi Lin and Hai-Lun Tu designed an $O(m+n)$ time algorithm for interval graphs and an $O(m(m+n))$ time algorithm for circular-arc graphs. They to solve the paired domination problem in interval graphs, They propose an $O(n)$ time algorithm that searches for a minimum paired-dominating set of $G$ incrementally in a greedy manner. Then they extend the results to design an algorithm for circular-arc graphs that also runs in $O(n)$ time\cite {lamport3}.









\bibliographystyle{plain}
\bibliography{Introduction.bib}



\end{document}
