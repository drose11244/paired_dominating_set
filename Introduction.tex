\documentclass[12pt]{article}

\usepackage{amsmath}
\usepackage{amssymb}
\usepackage{amsthm}
\usepackage{algorithmic}
\usepackage{enumerate}
\usepackage{float}[1]


%Gummi|065|=)
\title{\textbf{Finding paired-dominating sets }}
%\author{Alexander van der Mey \and 
%		Wei-Ning Huang \and
%		Dion Timmermann}

\date{}
\begin{document}

\maketitle
\newtheorem{lemma}{Lemma}

\section{Introduction}
A graph is an ordered pair $G=(V,E)$ consisting of a finite nonempty set $V$ of vertices and a set $E$ of edges, where each edge is an unordered pair of vertices. A dominating set of $G$ is a set $D \subseteq V$ such that each vertex not in $D$ has at least one neighbor in $D$. A paired-dominating set is a dominating set whose induced subgraph contains at least one perfect matching~\cite{1}.

Raz and Safra prove that the dominating set problem has no polynomial-time $(c \log|V|)$-approximation algorithms for some $c \textgreater  0$ unless $\text{P}=\text{NP}$~\cite{2}.

Lin and Tu design an $O(|E|+|V|)$-time algorithm for interval graphs and an $O(|E|(|E|+|V|))$-time algorithm for circular-arc graphs, for the minimum paired-dominating set problem~\cite {3}.

Let $f\colon \mathbb{N}\to \mathbb{N}$ be any function. If, given any graph $G=(V,E)$, an algorithm $A$ outputs a paired-dominating set of $G$ whose size is at most $f(|V|)$ times the minimum, then $A$ is said to be $f(|V|)$-approximate for the minimum paired-dominating set problem. 
\clearpage

\begin{figure}
\begin{algorithmic}[1]

\STATE ${\cal D}\leftarrow \emptyset$;
\WHILE{$\bigcup_{v\in {\cal D}}\,N[v]\neq V$}
  \STATE Among the edges in $E$ not having an endpoint in ${\cal D}$, pick
an edge $(a,b)$
that maximizes $|(N[a]\cup N[b])\cap (V\setminus \bigcup_{v\in {\cal D}}\,N[v])|$,
breaking ties arbitrarily;
  \STATE ${\cal D}\leftarrow {\cal D}\cup\{a,b\}$;
\ENDWHILE
\RETURN ${\cal D}$;
\end{algorithmic}
\caption{A greedy algorithm for finding paired-dominating sets}
\label{abc}
\end{figure}

%lemma 1
In the sequel, assume without loss of generality that $G$ is connected. The following lemma is a consequence of line~2 of \text{the algorithm in Fig.~\ref{abc}}.
\begin{lemma}
%During each iteration of the loop in lines~2--5 of $A$,
Whenever line~3 of the algorithm in Fig.~\ref{abc} is executed, 
$$
\left|V\setminus \bigcup_{v\in {\cal D}}\,N[v]\right|>0.
$$
\end{lemma}
Let $D^*$ be a smallest dominating set of $G=(V,E)$.

%lemma 2
\begin{lemma}
%During each iteration of the loop in lines~2--5 of $A$,
Whenever line~3 of the algorithm in Fig.~\ref{abc} is executed,
there exists $u\in D^*\setminus {\cal D}$ satisfying

\begin{equation}
\left|N[u]\cap \left(V\setminus \bigcup_{v\in {\cal D}}\,N[v]\right)\right|
\ge \frac{1}{|D^*|}
\cdot\left|V\setminus \bigcup_{v\in {\cal D}}\,N[v]\right|
\label{lemma2-equation}
\end{equation}
and that $N(u)\not\subseteq {\cal D}$.
\end{lemma}

\begin{proof}
Recall that $D^*$ is a dominating set. So the union of $N[v]$ over all $v$ $\in$ $D^*$ equals $V, $ which is a superset of $V\setminus \bigcup_{v\in{\cal D}} N[v]$. Therefore, there exists $u\in D^*$ such that $N[u]$ covers at least a $\frac{1}{|D^*|}$ proportion of $V\setminus \bigcup_{v\in{\cal D}} N[v]$, implying inequality $(1)$. We have $u\notin {\cal D}$ for, otherwise, the left-hand side of inequality $(1)$ vanishes, contradicting Lemma 1.
If $N(u)\subseteq {\cal D}$,  then $N[u]\subseteq \bigcup_{v\in {\cal D}} N[v]$, and therefore the left-hand side of inequality $(1)$ is 0, a contradiction to Lemma 1.
\end{proof}



\clearpage

%lemma 3
\begin{lemma}
%During each iteration of the loop in lines~2--5 of $A$,
Right after each execution of line~3 of the algorithm in Fig.~\ref{abc},
\begin{equation}
\left|\left(N[a]\cup N[b]\right)\cap \left(V\setminus \bigcup_{v\in {\cal D}}\,N[v]\right)\right|
\ge \frac{1}{|D^*|}
\cdot\left|V\setminus \bigcup_{v\in {\cal D}}\,N[v]\right|.
\end{equation}
\end{lemma}
\begin{proof}
Let $u\in D^*\setminus {\cal D}$ be as in Lemma~2. As $N(u)\not\subseteq {\cal D}$, we may pick a neighbor $u'\notin {\cal D}$ of $u$. By Lemma 2,

\begin{equation}
\left|\left(N[u]\cup N[u']\right)\cap \left(V\setminus \bigcup_{v\in {\cal D}}\,N[v]\right)\right|
\ge \frac{1}{|D^*|}
\cdot\left|V\setminus \bigcup_{v\in {\cal D}}\,N[v]\right|.
\end{equation}
By line 3 of the algorithm in Fig. 1, 
$$
\left|\left(N[a]\cup N[b]\right)\cap \left(V\setminus \bigcup_{v\in {\cal D}}\,N[v]\right)\right| \geq \left|\left(N[u]\cup N[u']\right)\cap \left(V\setminus \bigcup_{v\in {\cal D}}\,N[v]\right)\right|.
$$
Will find that the left side of the equation. Finally will begin Lemma~1 approaches zero to zero when the time is complete.
\end{proof}
%說一下上面這個式子和演算法第三行合起來,就可以得到(A)式喔
%lemma 4
\begin{lemma}
whenever lin4 of the algorithm in Fig.~\ref{abc} is executed, the (2) will became within $1-1/|D^*|$.
%這裡加一個lemma說一下,每次執行第四行都會使(2)變成原來的1-1/|D^*|倍以內

\end{lemma}


%lemma 5
\begin{lemma}
Given any graph $G$, the algorithm in Fig.~\ref{abc} outputs a paired-dominating set of $G$.
\end{lemma}

\bibliographystyle{plain}
\bibliography{Introduction}



\end{document}
