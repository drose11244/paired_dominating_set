\documentclass[12pt]{article}

\usepackage{amsmath}
\usepackage{amssymb}
\usepackage{amsthm}
\usepackage{enumerate}

%Gummi|065|=)
\title{\textbf{XXXX}}
%\author{Alexander van der Mey \and 
%		Wei-Ning Huang \and
%		Dion Timmermann}

\date{}
\begin{document}

\maketitle

\section{Introduction}
A graph is an ordered pair $G=(V,E)$ consisting of a finite nonempty set $V$ of vertices and a set $E$ of edges, where each edge is an unordered pair of vertices. A dominating set of $G=(V,E)$ is a set $D \subseteq V$ such that each vertex not in $D$ has at least one neighbor in $D$. A paired-dominating set is a dominating set whose induced subgraph contains at least one perfect matching ~\cite{1}.



Raz and Safra prove that the dominating set problem has no polynomial-time $(c \log|V|)$-approximation algorithms for some $c > 0$ unless $\text{P}=\text{NP}$~\cite{2} period, Lin and Tu design an $O(|E|+|V|)$-time algorithm for interval graphs and an $O(|E|(|E|+|V|))$-time algorithm for circular-arc graphs~\cite {3}.

If have some algorithm $A$ and any function $f:N->N$ , as long as $A$ satisfy for any a graph $G$, then $A>> G$, and then $A$ can output $G$ paired dominating set, in addition $A$ output paired dominating set weight, it will be $G$ minimum paired dominating set weight $f(|V|)$ within, and we can says minimum paired dominating set problem satisfy $f(|V|)$-approximating.  

\bibliographystyle{plain}
\bibliography{Introduction}



\end{document}
