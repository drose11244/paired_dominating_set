%\documentclass[pdf,nocolorBG,slideColor,blends]{prosper}
\documentclass{beamer}
%\usepackage{graphicx}
\usepackage{beamerthemesplit}
%\usepackage{beamerthemeepa}
%\usepackage{xcolor}
%\usepackage{natbib}
\usepackage{amssymb}
\usepackage{amsmath}
\usepackage{amsthm}
\usepackage{algorithmic}
%\usepackage{amsthm, amssymb, amsmath}


% for themes, etc.
\mode<presentation>
%{  \setbeamertemplate{background canvas}[vertical
%shading][bottom=red!10,top=blue!10]
%   \usetheme{Warsaw}
%   \usefonttheme[onlysmall]{structurebold}
%}

\renewcommand{\thefootnote}{\alph{footnote}}

\setbeamertemplate{footline}[page number]{}
\setbeamertemplate{navigation symbols}{}

%\usepackage{times}  % fonts are up to you
%\usepackage{graphicx}
%\usepackage{subfigure}
%\definecolor{blue2}{cmyk}{.94,.11,0,0}
% these will be used later in the title page

\newcommand{\comment}[1]{}

\usepackage{graphicx}
\begin{document}
\title{Finding paired-dominating sets}
\author{Yu-Sen Kao\\
\ \\
Joint work with Ching-Lueh Chang \\
\ \\
Yuan Ze University }


\date{2016/7/3}

%\maketitle
\setcounter{footnote}{0}
\begin{frame}
\titlepage
\end{frame}

%\section[Outline]{}

%\setcounter{footnote}{0}
%\begin{frame}\frametitle{Outline}
%\tableofcontents
%\end{frame}

%\section{Motivations}
%\section{The model}

%page 2 Defintions
\setcounter{footnote}{0}
\begin{frame}\frametitle{Definitions}
\begin{itemize}
\item  A dominating set of $G=(V, E)$ is a set $D \subseteq V$ such that each vertex not in $D$ has at least one neighbor in $D$. 
\item  A paired-dominating set is a dominating set whose induced subgraph contains at least one perfect matching. 
\end{itemize}
\end{frame}

%page 3 Related works
\setcounter{footnote}{0}
\begin{frame}\frametitle{Related works}
\begin{itemize}
\item  Haynes and Slater (1998) prove the NP-completeness of the minimum paired-dominating set problem. 
\item  Lin and Tu (2015) design an $O(|E|+|V|)$-time algorithm for interval graphs and an $O(|E|(|E|+|V|))$-time algorithm for circular-arc graphs. 
\end{itemize}
\end{frame}

%page 4 pseudocode
\begin{frame}\frametitle{Pseudocode}
%pseudo code has some problem.
\begin{algorithmic}[1]
\STATE ${\cal D}\leftarrow \emptyset$;
\WHILE{$\bigcup_{v\in {\cal D}}\,N[v]\neq V$}
  \STATE Among the edges in $E$ not having an endpoint in ${\cal D}$, pick
an edge $(a,b)$
that maximizes $|(N[a]\cup N[b])\cap (V\setminus \bigcup_{v\in {\cal D}}\,N[v])|$,
breaking ties arbitrarily;
  \STATE ${\cal D}\leftarrow {\cal D}\cup\{a,b\}$;
\ENDWHILE
\RETURN ${\cal D}$;\\
\ \\
Our algorithm is a modification of the famous approximation algorithm for set covering.
\end{algorithmic}
\end{frame}

%page 5 analysis-lemma 2
\setcounter{footnote}{0}
\begin{frame}\frametitle{Analysis}
\begin{lemma}
\ \\ 
Let $D^*$ be a dominating set of $G$.  Whenever line~3 of our algorithm is executed,  there exists $u$ $\in D^*\setminus {\cal D}$ such that   
%During each iteration of the loop in lines~2--5 of $A$,
\begin{eqnarray}
\left|N[u]\cap \left(V\setminus \bigcup_{v\in {\cal D}}\,N[v]\right)\right|
\ge \frac{1}{|D^*|}
\cdot\left|V\setminus \bigcup_{v\in {\cal D}}\,N[v]\right|
\label{thekeyinequality}
\end{eqnarray}
and that $N(u)\not\subseteq {\cal D}$.
\end{lemma}
\begin{proof}
%Clearly,
As $D^*$ is a dominating set,
{\footnotesize
$$
V\setminus \bigcup_{v\in {\cal D}}\,N[v]
\subseteq
\bigcup_{v\in D^*}\, N[v].$$
}
So by the averaging argument, there exists
$u\in D^*$ satisfying inequality~(\ref{thekeyinequality}).
%It is not hard to verify that
We have
$u\notin {\cal D}$ and $N(u)\not\subseteq {\cal D}$
for, otherwise, the left-hand side of inequality~(\ref{thekeyinequality})
would vanish.
\end{proof}
\end{frame}

%page 6 analysis-lemma 3-1
\setcounter{footnote}{0}
\begin{frame}\frametitle{Analysis}
\begin{corollary}
Right after each execution of line~3 of our algorithm,
\label{thekeycorollary}
%During each iteration of the loop in lines~2--5 of $A$,
$$
\left|\left(N[a]\cup N[b]\right)\cap \left(V\setminus \bigcup_{v\in {\cal D}}\,N[v]\right)\right|
\ge \frac{1}{|D^*|}
\cdot\left|V\setminus \bigcup_{v\in {\cal D}}\,N[v]\right|.
$$
\end{corollary}
%By elementary calculus, $(1-1/n)^n<1/e$ for all $n\in\mathbb{Z}^+$.
%Repeatedly invoking Corollary~\ref{thekeycorollary},
%and line~4 of the algorithm in Fig.~\ref{abc},
After $|D^*|\cdot\lceil\log|V|\rceil$ iterations,
$$
\left|\,
V\setminus \bigcup_{v\in {\cal D}}\,N[v]
\,\right|
\leq\left(1-\frac{1}{|D^*|}\right)^{|D^*|\cdot\lceil\log|V|\rceil}
|V|<1
$$
by repeatedly invoking the above
%corollary~\ref{thekeycorollary},
corollary,
implying
$$
\left|\, V\setminus \bigcup_{v\in {\cal D}}\,N[v] \,\right|=0.
$$
\end{frame}
%page 7 analysis-lemma 3-2
\setcounter{footnote}{0}
\begin{frame}\frametitle{}
\begin{center}
When the algorithm halts,
$|{\cal D}|$ is simply twice the number of iterations.
As a result, the algorithm outputs
a set ${\cal D}$ of
size at most $2\cdot|D^*|\cdot\lceil\log|V|\rceil$.
	

\end{center}
\end{frame}

%page 8 main result
\setcounter{footnote}{0}
\begin{frame}\frametitle{Main result}
\begin{theorem}
The minimum paired-dominating set problem
has a polynomial-time
$(2\cdot\lceil\log|V|\rceil)$-approximation algorithm.
\end{theorem}
\end{frame}


\setcounter{footnote}{0}
\begin{frame}\frametitle{}
\begin{center}
{\large Thank you!}
\end{center}
\end{frame}


%\bibliographystyle{alpha}
%\bibliography{presentation_proposal}

\end{document}






